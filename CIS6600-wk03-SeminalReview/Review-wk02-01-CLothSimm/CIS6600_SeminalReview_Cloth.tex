\documentclass[12pt]{article}

\usepackage[margin=1in]{geometry}
\usepackage{titlesec}
\usepackage{enumitem}
\usepackage{ulem}
\usepackage{fancyhdr}


\newcommand{\school}{
     \textbf{University of Pennsylvania}
}

\newcommand{\courseno}{
     \textbf{CIS 6600}
}

\newcommand{\paperTitle}{
    \textit{\normalsize{Large Steps in Cloth Simulation}}
}

% Title formatting
\titleformat{\section}{\large\bfseries}{\thesection.}{1em}{}

% Setup fancy headers
\pagestyle{fancy}
\fancyhf{} % clear all header and footer fields
\fancyhead[R]{\school{} \\ \courseno{}}
\renewcommand{\headrulewidth}{0.4pt} % add header rule

\fancypagestyle{plain}{
    \fancyhf{} % clear all header and footer fields
    \fancyhead[R]{\school{} \\ \courseno{}} % set the right header
    \renewcommand{\headrulewidth}{0.4pt} % add header rule
}

% Begin Document
\begin{document}

\title{\Large\uline{\textbf{Paper Review}} \\[0.4em]
\paperTitle{} 
}
\author{Yueyang Li}
\date{2024-02-07}

\maketitle

\section{Paper Title, Authors, and Affiliations}
\begin{itemize}[noitemsep]
    \item \textbf{Title}: Large Steps in Cloth Simulation
    \item \textbf{Authors}: David Baraff, Andrew Witkin
    \item \textbf{Affiliations}: Robotics Institute, Carnegie Mellon University (at publication time), later Pixar Animation Studios
\end{itemize}

\section{Main Contribution of the Paper}
The paper introduces a cloth simulation system capable of taking large, stable time steps, a significant improvement over previous methods that required small steps to prevent numerical instability. Its key contributions include:
\begin{itemize}[noitemsep]
    \item \textbf{Stable Large Time Steps for Cloth Simulation}: Enables the system to take large, stable time steps, addressing the limitation of traditional small-step methods.
    \item \textbf{Implicit Integration with Constraint Enforcement}: Combines implicit numerical integration with direct constraint enforcement, ensuring realistic cloth movements and stability.
    \item \textbf{Sparse Linear System Solved with Modified Conjugate Gradient}: Generates a large sparse linear system per time step, solved efficiently with a modified conjugate gradient method that integrates constraint enforcement without extra penalty forces.
    \item \textbf{Significant Speedup Over Prior Methods}: Achieves substantial speed improvements, making real-time or near-real-time animation of high-resolution cloth feasible.
\end{itemize}

\section{Outline of Major Topics}
\begin{enumerate}[noitemsep]
    \item \textbf{Background on Cloth Simulation}
    \begin{enumerate}[noitemsep]
         \item Discusses previous methods based on explicit integration (Euler, Runge-Kutta) and their limitations with stiff differential equations.
    \end{enumerate}
    \item \textbf{Implicit Integration Approach}
    \begin{enumerate}[noitemsep]
         \item Uses backward Euler integration to compute stable cloth movements with large time steps.
         \item Derives a linear system formulation that maintains realism in stretch, shear, and bending forces.
    \end{enumerate}
    \item \textbf{Handling Constraints Without Lagrange Multipliers}
    \begin{enumerate}[noitemsep]
         \item Introduces a method for direct enforcement of position and velocity constraints without additional penalty forces.
         \item The conjugate gradient solver ensures that constraints are exactly met regardless of the number of iterations.
    \end{enumerate}
    \item \textbf{Sparse System and Solver Efficiency}
    \begin{enumerate}[noitemsep]
         \item Modifies the conjugate gradient method to preserve symmetry and dynamically enforce constraints.
         \item Achieves rapid convergence even for large cloth meshes involving thousands of particles.
    \end{enumerate}
    \item \textbf{Adaptive Time Stepping}
    \begin{enumerate}[noitemsep]
         \item Step sizes adapt dynamically to prevent divergence.
         \item Monitors stiff stretch forces and reduces step sizes when instability is detected.
    \end{enumerate}
    \item \textbf{Collision Handling}
    \begin{enumerate}[noitemsep]
         \item Addresses cloth-cloth and cloth-solid interactions with bounding box collision detection.
         \item Uses spring-based repulsion for cloth-cloth interactions and directly modifies particle velocity for cloth-solid constraints.
    \end{enumerate}
    \item \textbf{Performance Results}
    \begin{enumerate}[noitemsep]
         \item Demonstrates simulations on garments with complex wrinkling and folding, applied to both keyframed and motion-captured characters.
         \item Reports substantial speedups compared to prior methods, with frame times ranging from approximately 2 to 14 seconds depending on resolution.
    \end{enumerate}
\end{enumerate}

\section{Two Things Liked or Found Interesting}
\begin{enumerate}[noitemsep]
    \item \textbf{Efficient Handling of Stiffness and Large Time Steps}: The paper resolves numerical instability with an elegant implicit integration approach, allowing for large, stable time steps and enabling high-resolution cloth simulation.
    \item \textbf{Constraint Enforcement Without Penalty Terms}: The direct enforcement of constraints improves accuracy and reduces computational overhead compared to traditional penalty-based methods, making the approach practical for real-time applications.
\end{enumerate}

\section{What Did You Not Like About the Paper?}
\begin{itemize}[noitemsep]
    \item \textbf{Assumption of Uniform Material Properties}: The paper does not thoroughly address material variations, as it largely focuses on stiffness adjustments. Real-world cloth often exhibits complex, non-uniform properties that may require further refinement.
    \item \textbf{Simplified Collision Handling}: The collision detection primarily relies on bounding box methods and simple repulsion forces, which might not capture the complexities of layered or interacting garments.
\end{itemize}

\section{Questions for the Authors}
\begin{enumerate}[noitemsep]
    \item Can the system efficiently handle complex layered clothing, such as garments with multiple interacting layers?
    \item Could an alternative solver, such as multigrid methods, further improve performance for large sparse systems in cloth simulation?
\end{enumerate}

\end{document}
