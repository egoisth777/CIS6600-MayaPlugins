
\documentclass[12pt]{article}

\usepackage[margin=1in]{geometry}
\usepackage{titlesec}
\usepackage{enumitem}
\usepackage{ulem}
\usepackage{fancyhdr}



\newcommand{\school}{
     \textbf{University of Pennsylvania}
}

\newcommand{\courseno}{
     \textbf{CIS 6600}
}

\newcommand{\paperTitle}{
    \textit{\normalsize{Procedural Modeling of Buildings}}  
}

% Title formatting
\titleformat{\section}{\large\bfseries}{\thesection.}{1em}{}

% Setup fancy headers
\pagestyle{fancy}
\fancyhf{} % clear all header and footer fields
\fancyhead[R]{\school{} \\ \courseno{}}
\renewcommand{\headrulewidth}{0.4pt} % add header rule

\fancypagestyle{plain}{
    \fancyhf{} % clear all header and footer fields
    \fancyhead[R]{\school{} \\ \courseno{}} % set the right header
    \renewcommand{\headrulewidth}{0.4pt} % add header rule
}

% Begin Document
\begin{document}

\title{\Large\uline{\textbf{Paper Review}} \\[0.4em]
\paperTitle{} 
}
\author{Yueyang Li}
\date{2024-01-30}

\maketitle

\section{Paper Title, Authors, and Affiliations}
\begin{itemize}[noitemsep]
    \item \textbf{Title}: Procedural Modeling of Buildings
    \item \textbf{Authors}: Pascal Müller, Peter Wonka, Simon Haegler, Andreas Ulmer, Luc Van Gool
    \item \textbf{Affiliations}: ETH Zürich, Arizona State University, K.U. Leuven
\end{itemize}

\section{Main Contribution of the Paper}
The paper introduces \textbf{CGA Shape}, a novel shape grammar for procedural modeling of buildings, addressing challenges like geometric consistency, complex shape configurations, and scalable urban models.

\section{Outline of Major Topics}
\begin{enumerate}[noitemsep]
    \item \textbf{Introduction}: Discusses procedural modeling and its benefits over traditional techniques.
    \item \textbf{Shape Grammar Design}: Details CGA Shape, production rules, and context-sensitive modeling strategies.
    \item \textbf{Applications}: Examples like Pompeii reconstruction and scalability demonstrations.
    \item \textbf{Discussion}: Comparison with existing methods and open problems.
\end{enumerate}

\section{Two Things Liked or Found Interesting}
\begin{enumerate}[noitemsep]
    \item Context-sensitive rules (e.g., occlusion, snapping) ensure realistic building designs.
    \item Scalability of the system for creating massive urban models with billions of polygons.
\end{enumerate}

\section{What Did You Not Like About the Paper?}
\begin{itemize}[noitemsep]
    \item Lack of focus on potential use cases in architectural design workflows.
    \item Steep learning curve for non-technical users.
\end{itemize}

\section{Questions for the Authors}
\begin{enumerate}[noitemsep]
    \item How does CGA Shape handle irregular building footprints in GIS data?
    \item Can the grammar dynamically adapt rules based on real-world constraints (e.g., material costs)?
\end{enumerate}

\end{document}
