\documentclass[12pt]{article}

\usepackage[margin=1in]{geometry}
\usepackage{titlesec}
\usepackage{enumitem}
\usepackage{ulem}
\usepackage{fancyhdr}


\newcommand{\school}{
     \textbf{University of Pennsylvania}
}

\newcommand{\courseno}{
     \textbf{CIS 6600}
}

\newcommand{\paperTitle}{
    \textit{\normalsize{Computer Facial Animation - A Survey}}
}

% Title formatting
\titleformat{\section}{\large\bfseries}{\thesection.}{1em}{}

% Setup fancy headers
\pagestyle{fancy}
\fancyhf{} % clear all header and footer fields
\fancyhead[R]{\school{} \\ \courseno{}}
\renewcommand{\headrulewidth}{0.4pt} % add header rule

\fancypagestyle{plain}{
    \fancyhf{} % clear all header and footer fields
    \fancyhead[R]{\school{} \\ \courseno{}} % set the right header
    \renewcommand{\headrulewidth}{0.4pt} % add header rule
}

% Begin Document
\begin{document}

\title{\Large\uline{\textbf{Paper Review}} \\[0.4em]
\paperTitle{} 
}
\author{Yueyang Li}
\date{2024-02-15}

\maketitle

\section{Paper Title, Authors, and Affiliations}
\begin{itemize}[noitemsep]
    \item \textbf{Title}: Computer Facial Animation - A Survey
    \item \textbf{Authors}: Zhigang Deng, Junyong Noh
    \item \textbf{Affiliations}:
    \begin{itemize}[noitemsep]
        \item Zhigang Deng: Computer Graphics and Interactive Media Lab, Department of Computer Science, University of Houston
        \item Junyong Noh: Graduate School of Culture Technology, Korea Advanced Institute of Science and Technology
    \end{itemize}
\end{itemize}

\section{Main Contribution of the Paper}
This survey paper provides a comprehensive overview of various techniques used in computer facial animation, ranging from classic methods like blend shapes and parameterizations to more advanced approaches such as physics-based muscle modeling and performance-driven animation. It also covers specific areas such as 3D face modeling, visual speech animation, and facial gesture generation, offering a broad perspective on the field and its challenges.

\section{Outline of Major Topics}
\begin{enumerate}[noitemsep]
    \item \textbf{Blend Shapes or Shape Interpolation}: Describes the fundamental technique of blend shapes, where facial expressions are created by linearly combining a set of predefined shapes.
    \item \textbf{Parameterizations}: Discusses parameterization techniques that offer more control over specific facial features and expressions.
    \item \textbf{Facial Action Coding System (FACS)}: Introduces FACS, a system for describing facial muscle movements and expressions using Action Units (AUs).
    \item \textbf{Deformation-Based Approaches}: Covers techniques that directly deform the facial mesh, including 2D and 3D morphing, Free-Form Deformation (FFD), and spline-based pseudo muscles.
    \item \textbf{Physics-Based Muscle Modeling}: Explores methods that simulate the behavior of facial muscles using mass-spring systems, vector representations, and layered spring meshes.
    \item \textbf{3D Face Modeling}: Discusses techniques for creating realistic 3D face models, including person-specific model creation and anthropometry.
    \item \textbf{Performance-Driven Facial Animation}: Reviews methods that use tracking data from human actors to drive facial animations.
    \item \textbf{MPEG-4 Facial Animation}: Describes the MPEG-4 standard for facial animation, which defines Face Definition Parameters (FDPs) and Facial Animation Parameters (FAPs).
    \item \textbf{Visual Speech Animation}: Discusses techniques for generating realistic lip movements synchronized with speech.
    \item \textbf{Facial Animation Editing}: Covers tools and techniques for editing facial animations to preserve the naturalness of expressions.
    \item \textbf{Facial Animation Transferring}: Discusses methods for transferring facial animations between different face models.
    \item \textbf{Facial Gesture Generation}: Covers techniques for generating realistic eye movements and head motions.
\end{enumerate}

\section{Things I Liked or Found Interesting}
\begin{enumerate}[noitemsep]
    \item The paper provides a clear and well-organized overview of the diverse techniques used in computer facial animation, making it a valuable resource for researchers and practitioners in the field.
    \item The inclusion of numerous figures and tables helps to illustrate the concepts and techniques discussed, making them easier to understand.
\end{enumerate}

\section{What Did You Not Like About the Paper?}
\begin{itemize}[noitemsep]
    \item The paper could benefit from a more in-depth discussion of the limitations and challenges associated with each technique.
    \item Some sections could be more detailed, particularly those on advanced topics like data-driven speech animation and facial motion transfer.
\end{itemize}

\section{Questions for the Authors}
\begin{enumerate}[noitemsep]
    \item What are the most promising avenues for future research in computer facial animation, particularly in addressing the challenges of real-time performance, automation, and adaptability to individual faces?
    \item How can the different techniques discussed in the paper be effectively combined to create more realistic and expressive facial animations?
\end{enumerate}
\end{document}
