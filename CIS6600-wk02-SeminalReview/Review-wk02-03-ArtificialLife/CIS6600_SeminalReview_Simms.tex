\documentclass[12pt]{article}

\usepackage[margin=1in]{geometry}
\usepackage{titlesec}
\usepackage{enumitem}
\usepackage{ulem}
\usepackage{fancyhdr}


\newcommand{\school}{
     \textbf{University of Pennsylvania}
}

\newcommand{\courseno}{
     \textbf{CIS 6600}
}

\newcommand{\paperTitle}{
    \textit{\normalsize{Evolving Virtual Creatures}}
}

% Title formatting
\titleformat{\section}{\large\bfseries}{\thesection.}{1em}{}

% Setup fancy headers
\pagestyle{fancy}
\fancyhf{} % clear all header and footer fields
\fancyhead[R]{\school{} \\ \courseno{}}
\renewcommand{\headrulewidth}{0.4pt} % add header rule

\fancypagestyle{plain}{
    \fancyhf{} % clear all header and footer fields
    \fancyhead[R]{\school{} \\ \courseno{}} % set the right header
    \renewcommand{\headrulewidth}{0.4pt} % add header rule
}

% Begin Document
\begin{document}

\title{\Large\uline{\textbf{Paper Review}} \\[0.4em]
\paperTitle{} 
}
\author{Karl Sims}
\date{2024-01-30}

\maketitle

\section{Paper Title, Authors, and Affiliations}
\begin{itemize}[noitemsep]
    \item \textbf{Title}: Evolving Virtual Creatures
    \item \textbf{Published in}: Computer Graphics, Annual Conference Series (SIGGRAPH '94 Proceedings), July 1994, pp.15-22.
    \item \textbf{Author}: Karl Sims
    \item \textbf{Affiliation}: Thinking Machines Corporation
\end{itemize}

\section{Main Contribution of the Paper}
This paper presents a novel system for evolving virtual creatures that move and behave in simulated 3D physical worlds. The key contributions include:
\begin{itemize}[noitemsep]
    \item The automatic generation of both morphology and control systems of virtual creatures using genetic algorithms.
    \item A directed graph-based genetic representation for encoding creature structures and neural networks.
    \item The co-evolution of morphology and control, allowing creatures to adapt their body structure and nervous system together.
    \item Demonstration of emergent locomotion behaviors such as swimming, walking, jumping, and light-following, without manually designing these behaviors.
    \item Application of evolutionary algorithms in procedural animation and artificial intelligence.
\end{itemize}

\section{Outline of Major Topics and Techniques}
\begin{enumerate}[noitemsep]
    \item \textbf{Introduction}
    \begin{itemize}[noitemsep]
        \item Trade-off between complexity and control:
        \begin{itemize}[noitemsep]
            \item Directly animating virtual entities allows precise control but is tedious.
            \item Dynamic simulation makes motions look natural but is difficult to control.
        \end{itemize}
        \item Optimization techniques, especially genetic algorithms, allow automated complexity generation.
        \item Prior work focused on evolving control systems for fixed structures, whereas this work evolves both morphology and control simultaneously.
    \end{itemize}
    \item \textbf{Creature Morphology Representation}
    \begin{itemize}[noitemsep]
        \item Creatures are represented as hierarchical articulated 3D structures.
        \item Morphology is encoded using a directed graph where:
        \begin{itemize}[noitemsep]
            \item Nodes represent body parts.
            \item Connections represent joints and attachment relationships.
        \end{itemize}
        \item The graph allows recursion, enabling fractal-like repeated structures.
    \end{itemize}
    \item \textbf{Creature Control – Neural Network Representation}
    \begin{itemize}[noitemsep]
        \item A virtual brain is evolved alongside morphology.
        \item The control system consists of:
        \begin{itemize}[noitemsep]
            \item Sensors (joint angles, contact sensors, photosensors).
            \item Neurons (processing units performing mathematical operations).
            \item Effectors (muscle forces applied to joints).
        \end{itemize}
        \item The neural network functions more like a dataflow system rather than a typical biological neural network.
        \item Oscillatory neurons allow rhythmic motion patterns, crucial for locomotion.
    \end{itemize}
    \item \textbf{Physical Simulation}
    \begin{itemize}[noitemsep]
        \item Rigid-body dynamics with articulated parts.
        \item Collision detection and response: Uses bounding box hierarchies to optimize performance.
        \item Friction, elasticity, and fluid resistance: Creatures can move on land or water with different physics settings.
        \item Energy conservation is enforced to prevent exploits in the physics engine.
    \end{itemize}
    \item \textbf{Evolution and Behavior Selection}
    \begin{itemize}[noitemsep]
        \item Fitness evaluation guides evolution towards desired behaviors.
        \item Different fitness criteria lead to different behaviors:
        \begin{itemize}[noitemsep]
            \item Swimming: Distance traveled in a simulated water environment.
            \item Walking: Horizontal distance covered on land.
            \item Jumping: Maximum height reached.
            \item Light-following: Ability to track a moving light source.
        \end{itemize}
        \item Elimination of unfit creatures: Creatures with collisions, excessive parts, or unrealistic structures are discarded.
    \end{itemize}
    \item \textbf{Evolutionary Process}
    \begin{itemize}[noitemsep]
        \item Population-based approach: Each generation consists of 300 creatures, with the top 20\% surviving.
        \item Genetic operations:
        \begin{itemize}[noitemsep]
            \item Mutation: Randomly alters parameters, connections, or structure.
            \item Crossover: Combines parts from two parents.
            \item Grafting: Connects a node from one creature to another.
        \end{itemize}
        \item Parallel implementation: Runs on a Connection Machine CM-5, with each processor testing different individuals.
    \end{itemize}
    \item \textbf{Results}
    \begin{itemize}[noitemsep]
        \item Evolved creatures exhibit diverse locomotion strategies:
        \begin{itemize}[noitemsep]
            \item Swimming creatures: Snake-like undulations, tail wagging, flipper motions.
            \item Walking creatures: Bipedal or multi-legged locomotion, hopping behaviors.
            \item Jumping creatures: Spring-like appendages for high jumps.
            \item Light-following creatures: Steering mechanisms to track light sources.
        \end{itemize}
        \item Unexpected emergent behaviors: Some creatures evolved rocking movements or oscillations to move forward efficiently.
    \end{itemize}
    \item \textbf{Future Work}
    \begin{itemize}[noitemsep]
        \item More complex tasks: Evolve creatures to perform multiple behaviors.
        \item Competitive evolution: Introduce population-based selection pressures.
        \item Real-world robotics: Constrain morphologies to buildable robotic designs.
        \item Aesthetic selection: Allow users to guide evolution based on appearance.
    \end{itemize}
    \item \textbf{Conclusion}
    \begin{itemize}[noitemsep]
        \item Demonstrates that morphology and control can co-evolve.
        \item Provides a practical approach for generating autonomous virtual creatures.
        \item Suggests that evolutionary methods could play a role in developing intelligent behavior in virtual entities.
    \end{itemize}
\end{enumerate}

\section{Two Things I Liked or Found Interesting}
\begin{enumerate}[noitemsep]
    \item \textbf{Emergent Behaviors Without Manual Design}
    \begin{itemize}[noitemsep]
        \item The creatures developed naturalistic locomotion strategies without explicit programming. I especially adore the idea of applying the concept of evolution and fitness into artificial life generation, this combines the beauty of genetic algorithm with the complexity of natural selection.
    \end{itemize}
    \item \textbf{Directed Graph Representation for Morphology \& Neural Networks}
    \begin{itemize}[noitemsep]
        \item Using graph structures for both body morphology and neural control enables scalable and modular designs. Also, the approach allows recursive self-similarity, leading to biologically plausible body structures.
    \end{itemize}
\end{enumerate}

\section{What Did You Not Like About the Paper?}
\begin{itemize}[noitemsep]
    \item \textbf{Limited Complexity in Behaviors}
    \begin{itemize}[noitemsep]
        \item While locomotion was well-evolved, there were no high-level behaviors like predator-prey interactions or adaptive learning. I think this can be a great extension to the current work.
    \end{itemize}
    \item \textbf{Lack of Robustness in Neural Control}
    \begin{itemize}[noitemsep]
        \item The neural system is not biologically realistic and is more like a data-processing system.Rather, it is unclear if the evolved behaviors generalize well beyond specific tasks.
    \end{itemize}
\end{itemize}

\section{Questions for the Authors}
\begin{enumerate}[noitemsep]
    \item How could this system evolve creatures with more complex and adaptive behaviors?
    \item Could reinforcement learning be combined with genetic evolution to improve decision-making? This could be helpful in providing more biological-plausible lookings of the creatures that we might not be able to generate with current approach.
\end{enumerate}

\end{document}
