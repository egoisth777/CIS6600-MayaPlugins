\documentclass[12pt]{article}

\usepackage[margin=1in]{geometry}
\usepackage{titlesec}
\usepackage{enumitem}
\usepackage{ulem}
\usepackage{fancyhdr}

\newcommand{\school}{
     \textbf{University of Pennsylvania}
}

\newcommand{\courseno}{
     \textbf{CIS 6600}
}

\newcommand{\paperTitle}{
    \textit{\normalsize{Real-Time Fluid Dynamics for Games}}
}

% Title formatting
\titleformat{\section}{\large\bfseries}{\thesection.}{1em}{}

% Setup fancy headers
\pagestyle{fancy}
\fancyhf{} % clear all header and footer fields
\fancyhead[R]{\school{} \\ \courseno{}}
\renewcommand{\headrulewidth}{0.4pt} % add header rule

\fancypagestyle{plain}{
    \fancyhf{} % clear all header and footer fields
    \fancyhead[R]{\school{} \\ \courseno{}} % set the right header
    \renewcommand{\headrulewidth}{0.4pt} % add header rule
}

% Begin Document
\begin{document}

\title{\Large\uline{\textbf{Paper Review}} \\[0.4em]
\paperTitle{} 
}
\author{Jos Stam}
\date{2024-01-30}

\maketitle

\section{Paper Title, Authors, and Affiliations}
\begin{itemize}[noitemsep]
    \item \textbf{Title}: Real-Time Fluid Dynamics for Games
    \item \textbf{Published in}: GDC 2003
    \item \textbf{Author}: Jos Stam
    \item \textbf{Affiliation}: Alias | Wavefront (formerly at SGI)
\end{itemize}

\section{Main Contribution of the Paper}
This paper introduces a fluid simulation algorithm specifically designed for real-time applications in games. Its key strengths include:
\begin{itemize}[noitemsep]
    \item Stability and efficiency that enable realistic fluid effects on standard PC hardware.
    \item A novel approach that avoids numerical instabilities even when using larger time steps.
    \item A straightforward and concise C implementation.
    \item Practical applicability demonstrated by its adoption in MAYA Fluid Effects™.
\end{itemize}

\section{Outline of Major Topics and Techniques}
\begin{enumerate}[noitemsep]
    \item \textbf{Introduction}
    \begin{itemize}[noitemsep]
        \item Highlights the limitations of traditional particle systems and ad-hoc methods for realistic fluid simulation in games.
        \item Emphasizes the challenges of real-time computation with the Navier-Stokes equations.
    \end{itemize}
    \item \textbf{Physics of Fluids}
    \begin{itemize}[noitemsep]
        \item Provides a concise overview of fluid dynamics principles.
        \item Focuses on the Navier-Stokes equations and fundamental behaviors: advection, diffusion, and the influence of external forces.
    \end{itemize}
    \item \textbf{Discretization on a Grid}
    \begin{itemize}[noitemsep]
        \item Utilizes a discrete grid to represent the fluid, with properties stored at cell centers.
        \item Discusses the trade-off between grid resolution and computational cost, as well as the importance of boundary conditions.
    \end{itemize}
    \item \textbf{Stable Fluid Solver}
    \begin{itemize}[noitemsep]
        \item Details the three main steps of the solver:
        \begin{itemize}[noitemsep]
            \item \textbf{Density Transport (Advection)}: Employs semi-Lagrangian advection for stable movement of fluid density.
            \item \textbf{Velocity Update}: Incorporates diffusion and self-advection using iterative solvers (e.g., Gauss-Seidel relaxation).
            \item \textbf{Mass Conservation}: A projection step to enforce fluid incompressibility.
        \end{itemize}
    \end{itemize}
    \item \textbf{Implementation}
    \begin{itemize}[noitemsep]
        \item Describes a concise and efficient C implementation.
        \item Outlines key functions such as add\_source, diffuse, advect, and project.
        \item Highlights performance optimizations and the ability to use larger time steps.
    \end{itemize}
    \item \textbf{Extensions and Applications}
    \begin{itemize}[noitemsep]
        \item Explores potential extensions to interactive smoke, fire, and water simulations.
        \item Discusses possible adaptations for various hardware platforms.
    \end{itemize}
    \item \textbf{Real-World Applications}
    \begin{itemize}[noitemsep]
        \item Showcases the algorithm's impact through its use in MAYA Fluid Effects™ and real-time game engines.
        \item Provides performance benchmarks on standard PCs.
    \end{itemize}
\end{enumerate}

\section{Two Things I Liked or Found Interesting}
\begin{enumerate}[noitemsep]
    \item \textbf{Stability and Efficiency}
    \begin{itemize}[noitemsep]
        \item The algorithm's ability to remain stable with larger time steps is essential for real-time applications.
        \item The use of semi-Lagrangian advection and iterative solvers effectively prevents numerical instabilities.
    \end{itemize}
    \item \textbf{Minimalist Implementation}
    \begin{itemize}[noitemsep]
        \item The concise C implementation demonstrates the algorithm's elegance and ease of understanding.
        \item Its simplicity has contributed to its widespread adoption in various applications, such as MAYA Fluid Effects™.
    \end{itemize}
\end{enumerate}

\section{What Did You Not Like About the Paper?}
\begin{itemize}[noitemsep]
    \item \textbf{Limited Discussion of Accuracy}
    \begin{itemize}[noitemsep]
        \item While the paper emphasizes visual realism, it lacks a rigorous comparison to physically accurate fluid simulations or established CFD benchmarks.
        \item The impact of numerical dissipation on realism is not thoroughly addressed.
    \end{itemize}
    \item \textbf{No GPU Consideration}
    \begin{itemize}[noitemsep]
        \item The paper predates the widespread use of GPUs for general-purpose computation.
        \item A discussion of GPU acceleration and parallel processing would have been a valuable addition.
    \end{itemize}
\end{itemize}

\section{Questions for the Authors}
\begin{enumerate}[noitemsep]
    \item Could this method be effectively adapted for large-scale simulations, such as ocean currents, considering the computational demands and potential for accumulated error?
    \item How does this approach compare to modern deep-learning-based fluid simulation techniques, particularly in terms of accuracy, computational cost, and ease of implementation?
\end{enumerate}
\end{document}
